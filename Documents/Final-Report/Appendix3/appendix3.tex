\chapter{Code Examples}
\label{appendix:code}
\section{Combining slides}
Code for combing slides and questions into one array:
\begin{verbatim}
        for ($i=1; $i<=$total; $i++) {
            foreach($questions as $question) {
                //Simple check to speed things up
                //No need to check it again
                if ($question->position < $i){
                    continue;
                }
                if ($question->position == $i) {
                    $combined[] = $question;
                    continue;
                }
            }

            foreach($slides as $slide) {
                //Simple check to speed things up
                //No need to check it again
                if ($slide->position < $i){
                    continue;
                }
                if ($slide->position == $i) {
                    $combined[] = $slide;
                    continue;
                }
            }
        }
\end{verbatim}
\newpage

\section{Multi user test}
Test that showcases Dusks ability to use multiple browsers that are needed to test the WebSockets.
\begin{verbatim}
    /**
     * Test two users joining different sessions
     */
    public function testTwoUsersJoinDifferentSessions()
    {
        $user1 = factory(User::class)->create();
        $user2 = factory(User::class)->create();
        $quiz1 = factory(Quiz::class)->create(['user_id' => $user1->id]);
        $quiz2 = factory(Quiz::class)->create(['user_id' => $user2->id]);
        factory(Session::class)->create([
            'user_id' => $user1->id,
            'quiz_id' => $quiz1->id,
            'running' => true,
        ]);
        factory(Session::class)->create([
            'user_id' => $user2->id,
            'quiz_id' => $quiz2->id,
            'running' => true,
        ]);
        
        $this->browse(function ($first, $second) use ($user1, $user2, $quiz1, $quiz2) {
            $first->visit('/quiz/' . $user1->session->session_key)
                ->assertSee($quiz1->name)
                ->assertDontSee($quiz2->name);

            $second->visit('/quiz/' . $user2->session->session_key)
                ->assertSee($quiz2->name)
                ->assertDontSee($quiz1->name);
        });
    }
\end{verbatim}
\newpage

\section{Multi selection saving}
This is part of the function that saves answers to the database. It highlights the extra logic needed to deal with multiple selection answers:
\begin{verbatim}
        //If its an array, its a multi select question
        if (is_array($answer)) {
            foreach($answer as $a) {
                if (strlen($answerToUse) == 0) {
                    //If its the first item, we dont want a comma
                    $answerToUse = $a;
                } else {
                    $answerToUse = $answerToUse . ', ' . $a;
                }
            }
        } else {
            $answerToUse = $answer;
        }
\end{verbatim}
\newpage

\section{CSV downloader}
A rather large function that creates the CSV of results:
\begin{verbatim}
        //loop over each question and add the relevant data to the csv
        foreach ($questions as $question) {
            //Add the title of each question
            $answersArray[] = [$question->question_text];
            
            //Get all the answers from the session
            $answers = Answer::where('question', $question->position)
                ->where('session_id', $session->id)->get();

            //Get the answers and how many times they were answered
            //Loop over all of them and add them to an array, if 
            //the item already exists, increment
            $answerValues = [];
            foreach($answers as $answer) {
                if(array_key_exists($answer->answer, $answerValues)) {
                    $answerValues[$answer->answer]++;
                } else {
                    $answerValues[$answer->answer] = 1;
                }
            }
            
            //Now to add two arrays to the csv
            //The first is the keys, the answers themselves that 
            //people click
            //The second is the number of times each was answered
            //This is so each array goes on a line in the csv
            $keys = [];
            $values = [];
            foreach($answerValues as $key=>$value) {
                //So if its a multi selection question, 
                //need to change answer1, answer2
                //to the various answers, split the string 
                //and loop over replacing with
                //the actual names
                if (strpos($key, ',')) {
                    $multiSelectKeys = explode(', ', $key);
                    $answerNames = '';
                    foreach($multiSelectKeys as $multiKey) {
                        $answerNames .= $question[$multiKey] . ', ';
                    }
                    $answerNames = rtrim($answerNames, ', ');
                    $keys[] = $answerNames;
                    $values[] = $value;
                } else {
                    $keys[] = $question[$key];
                    $values[] = $value;
                }
            }
            $answersArray[] = $keys;
            $answersArray[] = $values;
        }

        Excel::create($quizName . "-results", function($excel) 
        	use ($answersArray, $quizName) {
            // Set the spreadsheet title, creator, and description
            $excel->setTitle($quizName);

            // Build the spreadsheet, passing in the payments array
            $excel->sheet('sheet1', function($sheet) use ($answersArray) {
                $sheet->fromArray($answersArray, null, 'A1', false, false);
            });
        })->download('csv');
\end{verbatim}
\newpage