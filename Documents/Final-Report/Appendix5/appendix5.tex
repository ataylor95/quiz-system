\chapter{User stories}
A list of stories was produced that act as the functional requirements of the system. The stories have a difficulty ranking associated with each item in relation to the other stories. 1 would be the easiest and 10 the hardest.

\section{Initial user stories}
\label{appendix:initial-stories}
Stories for the first part (only quizzes):
\begin{enumerate}
	\item (6) Admins can create quizzes via the website
	\item (4) Quizzes contain a variety of questions
	\item (7) Admins can login to the website and run a session with a quiz
	\item (8) Multiple different sessions can be run simultaneously
	\item (3) The Admin specifies which question is being run by clicking next/prev question etc
	\item (5) Sessions can be joined by users via the website
	\item (3) Up to 300 users should be able to join and answer questions
	\item (1) Users answer the question being displayed by the quiz
	\item (2) The Admin can see what percentage of users connected to the session have answered
	\item (4) The Admin can show the results of the question in a sensible format e.g. graph
	\item (2) Admin can then save results as CSV or XML
	\item (2) The Admin can load from saved file to display again
	\item (4) Users should not be able to submit their own answers by altering the HTML, as they did with Qwizom
\end{enumerate}
These stories fit the second part (streaming slides), though there are some stories from the first part that are applicable within this part:
\begin{enumerate}
	\item (1) The Admin creates slides in a slideshow editor (Most likely Microsoft Office Power Point)
	\item (2) The Admin can place question slides within the slides during creation
	\item (10) The Admin can stream these slides to a session
	\item (5) Users can join the session and follow the slides as they are used	
	\item (4) The specified question slides will act as questions for the users connected to the session
	\item (2) Results are handled in the same way as the first web only part
	\item (3) The Admin can embed HTML content in quiz slides during creation
\end{enumerate}
\newpage

\section{Final user stories}
\label{appendix:final-stories}
Due to the redesign of part two, all the stories fit into the same part, and do not need to be split up. This final list was refined over the development period:
\begin{enumerate}
	\item (6) "Admins can create quizzes via the website". This is then broken into these sub stories:
	\begin{enumerate}
		\item (3) Admins can log into a backend
		\item (2) They are presented a list of their quizzes
		\item (5) They can create a new quiz in the backend
		\item (3) They can edit an existing quiz they own
	\end{enumerate}
	\item (4) Quizzes contain a variety of questions
	\item (7) Admins can login to the website and run a session with a quiz
	\item (8) Multiple different sessions can be run simultaneously
	\item (3) The Admin specifies which question is being run by clicking next/prev question etc
	\item (5) Sessions can be joined by users via the website
	\item (3) Up to 300 users should be able to join and answer questions
	\item (1) Users answer the question being displayed by the quiz
	\item (2) The Admin can see what percentage of users connected to the session have answered
	\item (4) The Admin can show the results of the question in a sensible format e.g. graph
	\item (2) Admin can then save results as CSV or XML
	\item (2) The Admin can load from saved file to display again
	\item (4) Users should not be able to submit their own answers by altering the HTML, as they did with Qwizom
	\item (5) The site should be mobile responsive
	\item (2) Admins should be able to change their session key
	\item (1) The admin creates slides in their preferred editor and exports them as a PDF
	\item (4) The admin can upload these slides to a quiz they have created in the past
	\item (3) The admin can reorder the questions within the quiz to move them around the slides
	\item (2) When this quiz is run, it should render the slides as well as questions in the order specified
\end{enumerate}