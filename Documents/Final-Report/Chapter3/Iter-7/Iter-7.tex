\subsection{Story: Site should be mobile responsive}
\subsubsection{Analysis - Breakdown of Tasks}
\begin{itemize}
	\item Create standard media query sizes
	\item Create media query for quiz questions
	\item Media query for welcome page
\end{itemize}
\subsubsection{Design}
A quiz page should look like the design specified in iteration 4.
\subsubsection{Implementation}
Finding the media query sizes was simple, Bootstrap recommends some default sizes for phones, tablets etc. (TODO: cite this https://getbootstrap.com/css/\#grid-media-queries ) Then it was a simple task of using the Chrome developer tools to view the page in mobile view and change the various sizes of buttons and titles to be more mobile friendly.
\subsubsection{Testing}
This is not really possible to test with automated testing, but should be easy to test with users later in the project.
\newpage

\subsection{Part Two Re-Design}
TODO: Some stuff about redesigning with slides
\newpage

\subsection{Story: Admins can stream slides}
\subsubsection{Analysis - Breakdown of Tasks}
\begin{itemize}
	\item Upload pdf slides
	\item Convert these slides to images
	\item Save references to these slides in the database
\end{itemize}
\subsubsection{Design}
TODO
\subsubsection{Implementation}
Upload of slides with standard form https://laracasts.com/series/whats-new-in-laravel-5-3/episodes/12
Convert these using a library that uses Imagaick - https://github.com/spatie/pdf-to-image
Save to database - why? Also talk about the way slides deleted because of size constraints. 20 slides as images is about 15mb which adds up quickly with multiple lecturers and lectures for each one. Is saving to the db actually needed? Websockets send the quizid and img name so maybe not.

Need to use php artisan storage:link to create a /storage folder
\subsubsection{Testing}
\newpage
