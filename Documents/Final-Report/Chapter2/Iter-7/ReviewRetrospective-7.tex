\subsection{Review and restrospective}
\subsubsection{Review}
\begin{itemize}
	\item (5) Site should be mobile responsive
	\item (4) The admin can upload these slides to a quiz they have created in the past
	\item (3) The admin can reorder the questions within the quiz to move them around the slides
\end{itemize}
The difficulties associated with the stories are relatively accurate, and due to more time being spent on this iteration than others before means they could all be completed, even though some earlier iterations had less stories with lesser difficulties completed. Most of the difficulties for these stories came from the first one, which involved a lot of CSS, something with which the developer is not experienced in writing. Story two also had some difficulty associated with saving files, however this was well documented online and in the end rather simple.

After doing this work, including the incomplete story, a design decision has been called into question. Instead of saving the position of slides and questions within their prospective tables, it might make sense to have a table for the position that then links to a relevant question or slide. It would make selecting all the data easier. This could be a potential future change.

Only one story was not fully completed: When the quiz is run, it should render the slides as well as question in the order specified. This is because only a tiny part is missing, which is that the images of slides it renders are somewhat out of position on the page, they just need some css and possible resizing before the story as a whole is finished.

\subsubsection{Retrospective}
This iteration really showcased the advantages of an iteration and story based workflow. The second part of the system needed a redesign due to the amount of time left and amount of work still required if the original idea was followed. Seeing as only some research had been done on the original idea, not much time has been lost moving over to the new design. This new design was also a lot smaller in scope, meaning that it was almost finished completed within the same iteration it was designed.

Something of note that is good is the previously written documentation. Due to a few of the stories in this iteration requiring changes to existing functions that were written a while ago, knowledge of how the system worked was lacking. Thanks to the PHPDoc and comments, it was easy to piece together how the system worked again.

Unfortunately, testing did not go that well, some was missed due to the rushing of development on the new design of part two.
\newpage
