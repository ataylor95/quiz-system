\subsection{Story: Admins can log into the website and run a session with a quiz}
\subsubsection{Analysis - Breakdown of Tasks}
After doing some spike work last week, this task can now be approached and broken into several sub tasks:
\begin{itemize}
	\item Set up config for Pusher
	\item Add event for broadcasting
	\item Write a command to trigger this event
	\item Add a button to quizzes to trigger this event
	\item Display this on the front end using the JavaScript which listens for Pusher events
	\item Add a session key box to front page
	\item Add session id to users
	\item Allows admins to change their id
	\item When admin clicks run, this id can be entered into the key box to join a channel with the name of the id
	\item The user will be presented with the initial quiz page which will be default filled with the name and description of the quiz
	\item The admin will see the same page but with an "admin panel"
	\item This admin panel has a next and previous button for questions
	\item When these buttons are pressed, the question is sent to pusher
	\item These question are rendered as a form on the user end and admin end
	\item The user can submit the form
\end{itemize}
\subsubsection{Design}
Mockups of the student end of the system: 
\begin{center}
	\begin{figure}
		\caption{Design for a quiz on a desktop}
		\includegraphics[width=\textwidth]{Chapter2/Iter-4/Quiz-Web-Design-Cropped}\\
		\label{fig:quiz-desktop}
	\end{figure}
	\vspace{1cm}
	\begin{figure}
		\caption{Design for a quiz on a mobile device}
		\includegraphics[scale=0.25]{Chapter2/Iter-4/Quiz-Mobile-Cropped}
		\label{fig:quiz-mobile}
	\end{figure}
\end{center}


\subsubsection{Implementation}
It began by configuring pusher, creating an event and writing some basic JavaScript to append to the front page. Some work was done on the admin panel as well, making it only visible to logged in users (the lecturer running the quiz), and adding the functionality for previous and next question buttons. These are buttons that send AJAX requests to quiz controller actions which then trigger the event for WebSockets with the appropriate question data.

Early into the iteration a major flaw with using the WebSockets came up, whilst new content was easy to add to the page, if a new user joined the session late, they would see a blank page/ or the original content of the page that has not yet been removed with JavaScript. To remedy this, the current position in the quiz should be kept track of and the PHP on the page should load the question specified at that position. At the same time, the WebSockets will be running and updating the page from that point onwards in the session. It was decided the best place for keeping track was within the session table, which now has appropriate columns for the position, quiz and if it is running.

Rendering the actual questions calls an action in the question controller, which takes the type of question and the quiz id and position. Using this data it renders one of several available views, one for each type of question like multiple choice or boolean. The views render a simple form showing the question, the possible answers as buttons and a submit button.

For a question to be rendered by default on page load as described above, the page includes the question by rendering the above view. However, it does not know the type of question to render, so it loops over the types and checks if the type of the question is equal to the ones defined earlier in the config file and renders the view if they are the same type. One problem with this is accessing the custom config file. The function to do this was inside the question controller and not the quiz controller. 

Whilst the function could be copied, it was better to create one single helper function that is global to the application. This involved creating a helpers file and adding the function there and then registering the helper function in the composer.json file\cite{laravel-helper-function}.

\subsubsection{Testing}
No testing as the story is not yet complete.
\newpage
