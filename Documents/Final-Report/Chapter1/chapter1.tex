\chapter{Background \& Objectives}
\section{Background} 
\subsection{Qwizdom}
Currently lecturers at Aberystwyth University have the option to use a service called Qwizdom (TODO cite) which allows lecturers to embed quiz questions into their slideshow presentations. During a lecture, students can join a session through an online portal and have the slides and questions streamed through to their devices. Quiz questions can then be answered by the students and their results can be displayed live by the lecturer, who can also save these results for later analysis.

A replacement system was wanted to fix some of the problems that Qwizdom has. Primarily that the University only has one session key, which means only one session can be used by all lecturers at any one time, leading to some clashes. With a new system, there would be no limit to the number of sessions that can be run. Other problems with Qwizdom include:
\begin{itemize}
	\item Maximum of 6 answers (TODO: was it six? ask Chris)
	\item Aging interface on the student end
	\item Students can submit their own answers by changing the HTML on the page
	\item Creating quizzes was tied into Microsoft PowerPoint which not all lecturers use
\end{itemize}

\section{Analysis}
\subsection{Two sections}
It was decided that the project would be split into two main parts, the first was to create an application that lecturers can run a simple quiz from without any slides. The second part involves developing an application or extension to the first part that allows the lecturer to create the quiz as part of a set of slides. The slide the lecturer is viewing would then be displayed in the quiz session on the web where students can access it. Once a relevant quiz slide appears, the students will be given the quiz options to select an answer in the same way as the first part. The lecturer can then display the results in the same way, though it would not be via the web application as before. This part of the project behaves in much the same was as Qwizdom.
The reason for having two parts is that the first would be mostly built as part of the development for the second part but if built as a standalone part it allows lecturers to create quizzes rather than a set of slides with quizzes within it. Features like the session joining, front-end view for students and the way answers are submitted would be used in the second part, which means only creating quizzes would need to be added for the first part for it to be stand alone. Additionally, the second part had the potential to be much larger in scope than originally anticipated and as such having the first part to extend in a different direction should the second part become unviable would be provide a safety net.
TODO: tense this
\subsection{Approach}
A web based approach was chosen for the first part. This application would allow lecturers to log in to an admin panel and from there create and run quizzes for the students. This would also lay the groundwork for the second part, setting up the front end for students, how sessions are run and how answering questions worked. 

The second part required more extensive research to be done when it was reached in development. But the main suggestion was to create an Microsoft PowerPoint extension much like Qwizdom to create slides embedded with questions. Due to the size of such an undertaking, another potential solution was to create the web application in such a way that an extension could be worked on in the future, i.e. set up the system for future extensions.
\subsubsection{Frameworks and hosting}

\section{Process}
The methodlolgy followed within this project was an Extreme Programming based approach. One of the core advantages of using an XP based approach is the ability to adapt to change. The second part of this project, implementing a method to stream slides to students, was much more open to change than the first part where the scope is far more understood. If another process such as Feature Driven Development was used, it would work for the first part of the project as all the requirements and features are known but most likely would be a struggle to use in the second part. XP gives the flexibility needed to complete the second part of the project whilst also allowing the first part to be done with ease.
\subsection{Practices}
A number of practices were selected for the project that work in a single developer project:
\begin{itemize}
	\item Test Driven Development - Writing tests before coding any of the application logic helps to enhance both the design and also mean tests should actually be written rather than left until the end and "hacked" in.
	\item Coding Standards - Keeping to strict coding standards helps ensure the code is good quality and easy to extend in the future as this project may well be.
	\item Small Releases - Small releases help enforce releases bit of working code regularly and results in a better overall project if the sub parts are all working.
	\item On Site Customer - This practice is somewhat applicable, the developer can act as a customer, and the project supervisor can also somewhat act as a customer, due to them being the originator of the idea and also a lecturer, one of the main users of this application.
	\item Merciless Refactoring - Refactoring is already encouraged by using TDD, but it can also be used at other points in the project to ensure the code is structured sensibly.
	\item Planning Game - This can be adapated to be done by one person at the start of the project, the list of stories was written and then iterations and releases planned out.
	\item Continuous Integration - Some online tools can be used to provide an CI workflow, to run tests continually with the constant small releases.
\end{itemize}
\subsection{Stories}

\section{Plan}