\chapter{Background \& Objectives}
\section{Background}
\subsection{Qwizdom}
\subsection{Extra requirements}

\section{Analysis}
\subsection{Two sections}
\subsection{Approach}
\subsubsection{Frameworks and hosting}
Taking into account the problem and what you learned from the background work, what was your analysis of the problem? How did your analysis help to decompose the problem into the main tasks that you would undertake? Were there alternative approaches? Why did you choose one approach compared to the alternatives? 

There should be a clear statement of the objectives of the work, which you will evaluate at the end of the work. 

In most cases, the agreed objectives or requirements will be the result of a compromise between what would ideally have been produced and what was determined to be possible in the time available. A discussion of the process of arriving at the final list is usually appropriate.

As mentioned in the lectures, think about possible security issues for the project topic. Whilst these might not be relevant for all projects, do consider if there are relevant for your project. Where there are relevant security issues, discuss how they will this affect the work that you are doing. Carry forward this discussion into relevant areas for design, implementation and testing.

\section{Process}
The methodlolgy followed within this project was an Extreme Programming based approach. One of the core advantages of using an XP based approach is the ability to adapt to change. The second part of this project, implementing a method to stream slides to students, was much more open to change than the first part where the scope is far more understood. If another process such as Feature Driven Development was used, it would work for the first part of the project as all the requirements and features are known but most likely would be a struggle to use in the second part. XP gives the flexibility needed to complete the second part of the project whilst also allowing the first part to be done with ease.
\subsection{Practices}
A number of practices were selected for the project that work in a single developer project:
\begin{itemize}
	\item Test Driven Development - Writing tests before coding any of the application logic helps to enhance both the design and also mean tests should actually be written rather than left until the end and "hacked" in.
	\item Coding Standards - Keeping to strict coding standards helps ensure the code is good quality and easy to extend in the future as this project may well be.
	\item Small Releases - Small releases help enforce releases bit of working code regularly and results in a better overall project if the sub parts are all working.
	\item On Site Customer - This practice is somewhat applicable, the developer can act as a customer, and the project supervisor can also somewhat act as a customer, due to them being the originator of the idea and also a lecturer, one of the main users of this application.
	\item Merciless Refactoring - Refactoring is already encouraged by using TDD, but it can also be used at other points in the project to ensure the code is structured sensibly.
	\item Planning Game - This can be adapated to be done by one person at the start of the project, the list of stories was written and then iterations and releases planned out.
	\item Continuous Integration - Some online tools can be used to provide an CI workflow, to run tests continually with the constant small releases.
\end{itemize}
\subsection{Stories}

\section{Plan}