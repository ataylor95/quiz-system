\thispagestyle{empty}

\begin{center}
    {\LARGE\bf Abstract}
\end{center}

The focus of this project was to create an application that can be used by lecturers to allow students to answer questions and display these results live in lectures. The project was based on a pre-existing application, called Qwizdom.

The system is a web based application built in Laravel, a PHP based framework. The system provides two main functions, the running of questions on their own, and streaming lecture slides with questions embedded within the slides to students. This system improves upon Qwizdom in several ways. It runs via a website rather than via a PowerPoint extension, meaning more lecturers can use it. Qwizdom also has security flaws, it allowed students to change the HTML of a the page and submit their own answers to be displayed by the lecturer.

Development followed an Extreme Programming approach, utilising a number of practices to help development. Stories were created and used as the functional requirements for the system, and stories were worked on individually within weekly iterations. Analysis, design, implementation and testing were done for each story rather than having an upfront design stage or end testing stage.

TODO: mention user testing when thats done

TODO: summary of eval when written?