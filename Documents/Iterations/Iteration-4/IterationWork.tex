\documentclass{article}

\usepackage{cite}
\usepackage{fancyhdr}

\pagestyle{fancy}
\lhead{Alex Taylor}
\cfoot{}
\rfoot{\thepage}

\title{Iteration: 4 (16/03 - 22/03)}
\author{Alex Taylor - amt22@aber.ac.uk}

\begin{document}

\maketitle
\begin{center}
	Version 0.2 (Draft)
\end{center}
\tableofcontents
\thispagestyle{empty}
\newpage

\section{Story: Admins can log into the website and run a session with a quiz}
\subsection{Analysis - Breakdown of Tasks}
After doing some spike work last week, this task can now be approached and broken into several sub tasks:
\begin{itemize}
	\item Set up config for Pusher
	\item Add event for broadcasting
	\item Write a command to trigger this event
	\item Add a button to quizzes to trigger this event
	\item Display this on the front end using the javascript which listens for Pusher events
	\item Add a session key box to front page
	\item Add session id to users
	\item Allows admins to change their id
	\item When admin clicks run, this id can be enetered into the key box to join a channel with the name of the id
	\item The user will be presented with the initial quiz page which will be default filled with the name and description of the quiz
	\item The admin will see the same page but with an "admin panel"
	\item This admin panel has a next and previous button for questions
	\item When these buttons are pressed, the question is sent to pusher
	\item These question are rendered as a form on the user end and admin end
	\item The user can submit the form
\end{itemize}
\subsection{Design}
Here are some mockups of the user end of the system:
\subsection{Implementation}
\subsection{Testing}
\subsection{Extras}
\newpage

%\section{Bibliography}
%\bibliographystyle{IEEEannotU}
%\bibliography{IEEEabrv,FrameworkHostingBib}

\end{document}