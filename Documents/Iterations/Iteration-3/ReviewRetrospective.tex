\documentclass{article}

\usepackage{fancyhdr}

\pagestyle{fancy}
\lhead{Alex Taylor}
\cfoot{}
\rfoot{\thepage}

\title{Iteration 3 - Review and Retrospective}
\author{Alex Taylor - amt22@aber.ac.uk}

\begin{document}

\maketitle
\begin{center}
	Version 1.0 (Release)
\end{center}
\thispagestyle{empty}
\newpage

\section{Review}
\subsection{Completed Stories}
None completed
\subsection{Incomplete Stories}
No actual story work was completed, however a large amount of non story work was completed. Additionally a large amount of spike work was completed for the story: Admins can login to the website and run a session with a quiz

There was a significant amount of refactoring in this iteration which was not too challenging. The most problematic bit was trying to get the database migrations done properly.

The spike work was particularly challenging this week because it was attempting to work with a new technology. It also involved the use of various parts of the Laravel framework not encountered before, like events and writing custom artisan commands.
\newpage

\section{Retrospective}
\subsection{What went well}
The spike work was very useful for the overrall devlopment of the application. Additionally the refactoring and restructing of the database and testing has helped a lot with the future outlook of the application.
\subsection{What went badly}
One bad thing was to do with the continuous integration on Travis. Laravel Dusk is a relatively new testing framework and Travis does not seem to work well with it. This means that the tests written cannot be run on Travis. It therefore makes CI somewhat pointless if the tests cannot be run and this will be left for now.

\end{document}