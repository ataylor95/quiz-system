\documentclass{article}

\usepackage{cite}
\usepackage{fancyhdr}

\pagestyle{fancy}
\lhead{Alex Taylor}
\cfoot{}
\rfoot{\thepage}

\title{Iteration: 9 (20/04 - 26/04)}
\author{Alex Taylor - amt22@aber.ac.uk}

\begin{document}

\maketitle
\begin{center}
	Version 1.0 (Release)
\end{center}
\tableofcontents
\thispagestyle{empty}
\newpage

\section{Report work}
A lot of report work was completed this last week. This primarily focussed on trying to get through the TODOs scattered throughout the report. It also included adding most of the citations for the system that were noted down during development. The testing and design chapters were also fleshed out far more, and once all this was completed a draft was sent to the project supervisor for some feedback.

\section{Security testing}
A number of tests were added to test the security of the system. This consisted of five tests for SQL injection and XSS across the site. Whilst Laravel is supposed to handle much of this it should still be tested to ensure the security of the system. There are only two main places that these attacks can take place, the session search field on the front page of the application, and the question and quiz creation pages in the admin area. The tests proved that Laravel does indeed stop XSS and SQL injection by escaping the inputs.

\section{Server setup}
There was some time spent on setting up the server for use in the user tests in the final iteration, the server was provided by the Department of Computer Science. The setup included moving the project onto the server via Git, setting up the MySQL server, and making the site visible to the internet with an htaccess.
\newpage

\section{Bug fixing}
There was a significant amount of bug fixing in this iteration. The main two were fixing the CSS of the slides in quizzes and multiple selection questions not submitting correctly. 

For the first, a lot of time was spent on trying to write custom CSS for the image tag that would centre it and fill the page. However, there was no actual need to write any CSS, instead the solution was to add a simple bootstrap container around the image. Whilst the image might be too high for the page, scrolling up and down is not much of a usability problem assuming most slides have blank space at the bottom. This Bootstrap container fixes the image sizing problems that were encountered beforehand, both on a standard desktop and on mobile and tablet devices automatically.

The second major bug fix was to how answers were saved for multiple selection questions. The difference between this and a boolean or multiple choice question is that multiple answers are submitted, and an array was chosen to do this. However, this array was not handled correctly when saving these results. The solution was to create entries of the string versions of the answers joined together, so an answer row would contain the answer of "answer1, answer3". TODO: code snippet of this stuff. Displaying these results did not require much changing except needing to split this string and loop over the items and replacing the "answer1" fields with the actual answer given in the database.
\newpage

%\section{Bibliography}
%\bibliographystyle{IEEEannotU}
%\bibliography{IEEEabrv,FrameworkHostingBib}

\end{document}