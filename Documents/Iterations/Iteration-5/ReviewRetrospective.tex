\documentclass{article}

\usepackage{fancyhdr}

\pagestyle{fancy}
\lhead{Alex Taylor}
\cfoot{}
\rfoot{\thepage}

\title{Iteration 5 - Review and Retrospective}
\author{Alex Taylor - amt22@aber.ac.uk}

\begin{document}

\maketitle
\begin{center}
	Version 1.0 (Release)
\end{center}
\tableofcontents
\thispagestyle{empty}
\newpage

\section{Review}
\subsection{Completed Stories}
This week seven stories have been completed:
\begin{enumerate}
	\item (7) Admins can login to the website and run a session with a quiz
	\item (8) Multiple different sessions can be run simultaneously
	\item (3) The Admin specifies which question is being run by clicking next/prev question etc
	\item (5) Sessions can be joined by users via the website
	\item (3) Up to 300 users should be able to join and answer questions
	\item (1) Users answer the question being displayed by the quiz
	\item (2) Admins should be able to change their session key
\end{enumerate}
The first six were completed at the end of three iterations of work, though the work was listed under the first stroy in that list, which means they some were undoubtedly finished before this iteration but not counted as being done. This was because of the large amount of cross over between the stories and it was decided all the work was to be completed at the same time.

In terms of difficulty, because of the way this worked on, breaking down difficulty is hard. However, the first one is relatively accurate, the second story  is not however. Once one channel was established, creating a second was trivial. Story 3 also really stands out as this is where a lot of work went during the last iteration, rendering the question was a large amount of work and also involved changing the database structure alongside the html, php and javascript work. 

Story 4 is also overestimated, joining a channel was relatively easy, so should probably be lowered to a difficulty of around 3. Number 5 is subjective, as Pusher the WebSocket service provider says it can support up to 100 users, though there has been no lrge scale test as of this moment. Story 6 should probably be increased as it was a bit more work than simply rendering a form, there is some javascript for submitting asnwers. The last story is quite accurate, even though it was a large amount of work, it was easy after having done a lot of similar work earlier in the project.
\subsection{Incomplete Stories}
None
\newpage

\section{Retrospective}
\subsection{What went well}
The actual work went really well, with a lot of completed stories the project is in a much better place now.
\subsection{What went badly}
Actually breaking all the work into their respective stories was not done, and this could have been used to track progress much better. Also it seems as though a lot of the predicted difficulties for the stories were off, both above and below estimate.

\end{document}