\documentclass{article}

\usepackage{fancyhdr}

\pagestyle{fancy}
\lhead{Alex Taylor}
\cfoot{}
\rfoot{\thepage}

\title{Iteration 1 - Review and Retrospective}
\author{Alex Taylor - amt22@aber.ac.uk}

\begin{document}

\maketitle
\begin{center}
	Version 1.0 (Release)
\end{center}
\thispagestyle{empty}
\newpage

\section{Review}
\subsection{Completed Stories}
\begin{enumerate}
	\item (6) Admins can create quizzes via the website
	\item (3) Admins can login to a backend
	\item (2) They are presented a list of their quizzes
\end{enumerate}
The first story was quite large so was split up, thus making it quite an easy task with no problems.

The second and third were not that challenging either, 2 in particular was made much easier thanks to the built in auth builder from Laravel. This means that the rating of 3 is wrong and it should be lower at a 1. Story 3 fits a difficulty of 2 as it provided no major problems, but it did give ample opportunity to learn Laravel much better. For example learning the MVC structure a bit more and how to use helper functions such as the one to get the currently authenticated user.
\subsection{Incomplete Stories}
All implementation tasks were completed, but some documentation for story 2 and 3 was missed. That is to be completed at the start of the next iteration.
\newpage

\section{Retrospective}
\subsection{What went well}
Breaking the stories up into substories and then those stories into subtasks is very useful for this project. It splits the functionality well allowing individual components to be built and showcased on the development site.
\subsection{What went badly}
The documentation did not go well, a majority of the design, testing and implementation documentation was missed during the iteration. This was mostly due to overestimating the amount of time available to write this. In further iterations, more emphasis should be placed the documentation section of work.

\end{document}