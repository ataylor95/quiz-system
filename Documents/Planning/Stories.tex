\documentclass{article}

\usepackage{fancyhdr}

\pagestyle{fancy}
\lhead{Alex Taylor}
\cfoot{}
\rfoot{\thepage}

\title{Stories}
\author{Alex Taylor - amt22@aber.ac.uk}

\begin{document}

\maketitle
\begin{center}
	Version 0.6 (Draft)
\end{center}
\thispagestyle{empty}

\section{Initial Stories}
Below is a list of stories for the system, alongside them is a difficulty ranking associated with each item in relation to the others. 1 would the easiest and 10 the hardest.
These stories fit the first part of the project:
\begin{itemize}
	\item (6) Admins can create quizzes via the website
	\item (4) Quizzes contain a variety of questions
	\item (7) Admins can login to the website and run a session with a quiz
	\item (8) Multiple different sessions can be run simultaneously
	\item (3) The Admin specifies which question is being run by clicking next/prev question etc
	\item (5) Sessions can be joined by users via the website
	\item (3) Up to 300 users should be able to join and answer questions
	\item (1) Users answer the question being displayed by the quiz
	\item (2) The Admin can see what percentage of users connected to the session have answered
	\item (4) The Admin can show the results of the question in a sensible format e.g. graph
	\item (2) Admin can then save results as CSV or XML
	\item (2) The Admin can load from saved file to display again
	\item (4) Users should not be able to submit their own answers by altering the HTML, as they did with Qwizom
\end{itemize}
These ones fit the second part, though there are some stories from the first part that are applicable within this part.
\begin{itemize}
	\item (1) The Admin creates slides in a slideshow editor (Most likely Microsoft Office Power Point)
	\item (2) The Admin can place question slides within the slides during creation
	\item (10) The Admin can stream these slides to a session
	\item (5) Users can join the session and follow the slides as they are used	
	\item (4) The specified question slides will act as questions for the users connected to the session
	\item (2) Results are handled in the same way as the first web only part
	\item (3) The Admin can embed HTML content in quiz slides during creation
\end{itemize}
\newpage

\section{Sub Stories}
\textbf{First Story}, "Admins can create Quizzes via the website" can be broken down into a number of sub stories:
\begin{itemize}
	\item (3) Admins can log into a backend
	\item (2) They are presented a list of their quizzes
	\item (5) They can create a new quiz in the backend
	\item (3) They can edit an existing quiz they own
\end{itemize}

\section{New Stories}
As within any project, the initial analysis of requirements might need to be added to or expanded upon. Therefore new stories can sometimes be added in, and thanks to XP these can be accounted for and fitted into development.

\textbf{Iteration 2:}
\begin{itemize}
	\item (4) A "super" admin user should have to approve new accounts being registered
	\item (2) A "super" admin user should be able to view all the users, quizzes and questions on the system and be able to edit and delete at will
	\item (5) The site should be mobile responsive
\end{itemize}

\end{document}